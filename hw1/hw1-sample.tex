\documentclass{article}
\usepackage[left=3cm,top=3cm,right=4cm,nohead,bottom=3cm]{geometry}
%THIS FILE IS AUTOMATICALLY GENERATED. DO NOT EDIT.
\usepackage{amsmath}
\usepackage{amssymb}
\usepackage{proof}
\usepackage{hyperref}
\newtheorem{lemma}{Lemma}
\newtheorem{corr}{Corollary}
\newenvironment{hint}{\noindent{\bf (Hint)}}{}
\newtheorem{thm}{Theorem}
\newtheorem{definition}{Definition}
\newtheorem{task}{Task}
\newenvironment{impltask}{\begin{task}}{\end{task}}
\newtheorem{bonus}{Bonus Task}

\newif\ifsol

\newcommand{\makehwtitle}[4]
{{\centering\Large\textbf{Homework #1: #2}
    \normalsize\\
15-897: Parallel Computing: Theory and Practice\\
Spring 2016\\
TA: Stefan Muller (smuller@cs.cmu.edu)\\
\ifsol \solcolor \textbf{SOLUTIONS} \normalcolor \\ \fi \vspace{0.5cm}%
Out: #3\\
Due: #4\\
}}

\newenvironment{proof}{\trivlist \item[\hskip \labelsep{\bf
Proof:}]}{\hfill$\Box$ \endtrivlist}

\newenvironment{sol}{
\trivlist \item[\hskip \labelsep{\bf
Solution:}]}{\endtrivlist}

\newcommand{\solution}[1]{\ifsol \begin{sol} #1 \end{sol} \fi}

\newcommand{\taskpts}[1]{[#1]}

\newenvironment{includeintemplate}{}{}
\newenvironment{noinclude}{}{}
\newcommand{\rulename}[1]{\text{\textsc{(#1)}}}
\newcommand{\einfer}[3]
        {\begin{equation}\label{eq:#1}\infer[]{#2}{#3}\end{equation}}

\allowdisplaybreaks       %mildly permissible to break up displayed equations


\title{15-897 Homework 1}
\author{YOUR NAME HERE}

\begin{document}
\maketitle

\section{Logistics}
\section{Submission}
\section{Theory}
\subsection{Be greedy}
\begin{task} \taskpts{15}
Show that this bound is tight by describing an algorithm which takes a span
$S \geq 2$
and a number of processors $P \geq 2$ and constructs a dag for which there
exists a greedy schedule of length {\em exactly} $\frac{W}{P} + S\frac{P-1}{P}$.
(In other words, you will be giving a family of dags, one for each $S$ and $P$.)
Using diagrams to help explain your algorithm will likely be helpful.

\textbf{Describe the schedule} which takes time $\frac{W}{P} + S\frac{P-1}{P}$
(see hint below).
\end{task}
\begin{sol}
\end{sol}

\subsection{Toward practical schedulers}
\begin{task} \taskpts{5}
Suppose we use this scheduler to compute
\texttt{fib(42)} (the $42^{nd}$ Fibonacci number) using the simple,
naturally parallel and recursive exponential-work algorithm.
How many vertices are there in the dag at level 20 (count the initial vertex
as level 0), assuming we do not do any granularity control? That is, how many
vertices will need to be stored in the container at this level?
Approximately how does this number scale at lower levels of the dag? What
does this mean for our scheduler's space usage?
\end{task}
\begin{sol}
\end{sol}

\begin{task} \taskpts{5}
Describe another likely problem (not having to do with space usage)
with our scheduler. Note that the description of the implementation was
deliberately vague. You may make whatever (reasonable) assumptions you'd like
about the exact implementation or the underlying system, but please document
whatever important and non-obvious assumptions you make so we know how
to interpret your answer.
\end{task}
\begin{sol}
\end{sol}

\begin{task} \taskpts{5}
Suggest an improvement to the scheduler which has better space usage. You do not
need to describe its implementation in detail, or to prove any properties of
it. This is just to get you thinking about scheduling approaches. Your scheduler
also need not solve the problem you posed in the previous task (though it would
be great if it did!)
\end{task}
\begin{sol}
\end{sol}

\section{Practice}
\subsection{Setting up PASL}
\subsection{Sequence Library}
\subsection{Merge}
\subsection{The Pittsburgh Skyline Problem (with thanks and apologies to 15-210)}
\subsection{Testing}
\subsection{Evaluation}
\begin{task} \taskpts{5}
Briefly explain what the speedup curve says about the parallelism of your
skyline code. If your curve doesn't look as good as you had hoped, suggest
why this might be the case.
\end{task}
\begin{sol}
\end{sol}

\end{document}